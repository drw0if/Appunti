\section{Primo programma}
Vogliamo fare un programma che esegue la divisione tra un numero a 16 bit ed uno a 8 bit:
\begin{verbatim}
    .include "m32def.inc"   ; includiamo i simboli messi
                            ; a disposizione da ATMEL
    .equ dividendo=1000
    .equ divisore=150

    .org 0x2a               ; lascio spazio vuoto fino a 0x2a
    
    start:
        ldi r16, low(RAMEND)
        ldi r17, high(RAMEND)
        out SPL, r16
        out SPH, r17
        ; stack inizializzato
        
        ldi XL, low(dividendo)
        ldi XH, high(dividendo)
        ldi YL, low(divisore)
        clr YH
        
        rcall divb
    
    fine:
        rjmp fine
        ; program halt
    
    ; X:    dividendo
    ; YL:   divisore
    ; Z:    quoziente
    ; divisione attraverso sottrazioni successive
    divb:
        tst YL
        brne ok
        sec         ; setto il carry
                    ; per segnalare errore
        ret
    
    ok:
        clr ZH
        clr ZL
        clr r0
    
    test:
        cp XL, YL
        cpc XH, r0  ; se ho carry ho finito
        brcc avanti
        clc
        ret
    
    avanti:
        sub XL, YL
        sbc XH, r0
        adiw ZL, 1
        rjmp test
\end{verbatim}
Compilando il programma si ottengono:
\begin{itemize}
    \item .hex: codice da flashare ul uC
    \item .lss: listato del programma compilato
    \item .map: contiene i valori dei simboli, sono numeri rappresentati su 32 bit ma in realtà durante l' assemblaggio vengono troncati a 16
\end{itemize}

Si ricordi che il file ottenuto dall' assemblaggio va flashato sulla FLASH interna del uC, per fare ciò bisogna cancellare tutto il suo contenuto, questo non fa altro che settare ad 1 tutta la flash, dopodiché si scrivono gli 0.
Questo si fa perché le memorie flash permettono di passare un singolo bit da 1 a 0 ma non viceversa, per fare ciò si deve cancellare tutto quanto, resettare la flash, quindi in genere non si fa a runtime una scrittura in flash.
Inoltre le flash hanno un numero massimo di riscritture, in particolare l' ATmega32 garantisce fino a 10k scritture.


