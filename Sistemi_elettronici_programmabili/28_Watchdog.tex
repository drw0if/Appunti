\section{Watchdog}
E' un timer simile al timer 0 usato per una feature di sicurezza.
Ha un contatore interno che al wrap causa il reset del chip.
Il software deve quindi azzerare il contatore periodicamente via software in modo da evitare il reset.

Utilizza un oscillatore di clock tutto suo sempre per motivi di sicurezza, usa però un circuito LC a bassa precisione.
Contiene un prescaler programmabile.

Si programma tramite il registro WDTCR:
\begin{figure}[H]
    \centering
    \includegraphics[width=330px]{images/28_Watchdog/WDTCR.png}
\end{figure}
\begin{itemize}
    \item WDE: fa partire il watchdog
    \item WDTOE: va settato ad 1 per poter disabilitare il watchdog.
    Subito dopo averlo settato si hanno 4 cicli di clock per scrivere 0 su WDE
    \item WDP0-2: usati per settare il prescaler 
\end{itemize}

Per resettare il contatore si usa l' istruzione \verb{wdr{.
