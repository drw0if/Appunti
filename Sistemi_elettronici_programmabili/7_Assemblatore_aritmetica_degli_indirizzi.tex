\section{Funzionamento dell'assemblatore e aritmetica degli indirizzi}
Il software si scrive utilizzando un comune editor di testo, il codice sorgente non è nient' altro che testo, per trasformarlo in codice macchina comprensibile dal microcontrollore si deve utilizzare un programma chiamato \emph{assemblatore}.
Questo software si occupa di prendere il testo, leggerlo, trasformare il sorgente in codice macchina e costruire un file \emph{binario} pronto per essere scritto sulla memoria FLASH del SoC.

\subsection{Location counter}
Per eseguire la traduzione l'assemblatore ha la necessità di conoscere l' indirizzo di ogni istruzione, per fare ciò una variabile chiamata \emph{location\_counter} viene inizializzata a zero e man mano che le istruzioni del codice sorgente vengono elaborate viene incrementato (di 1 o 2 in base alla lunghezza dell'istruzione costruita).
\begin{verbatim}
    LDI r9, 47      # address: 0, size: 1
    ADIW r10, 810   # address: 1, size: 2
    MOV r0, r3      # address: 3, size: 1
\end{verbatim}

Attraverso alcune direttive all'assemblatore il location\_counter si può modificare, questo è spesso utilizzato per lasciare buchi nella flash.

\subsection{Simboli}
Un simbolo è una stringa di testo usata come identificatore per un valore.
Si possono usare lettere e numeri (non come primo carattere), sono case-insensitive, gli underscore (\_) sono ammessi.
L'assemblatore associa a ciascun simbolo un valore a 16 bit, si costruisce una tabella man mano che si va avanti e si incontrano dichiarazioni di simboli.
Ci sono due metodi per dichiarare un simbolo:
\begin{itemize}
    \item direttive .EQU e .DEF:
    \begin{verbatim}
        .EQU FOO=0x3900
        .DEF BAR=0x03A0
    \end{verbatim}
    entrambe queste direttive associano il valore a destra dell' uguale alla stringa a sinistra dell' uguale
    
    \item etichette:
    \begin{verbatim}
        QUI: <istruzione>
    \end{verbatim}
    si associa l' indirizzo dell' istruzione (il valore del location\_counter) all' etichetta specificata
\end{itemize}

\subsubsection{Etichette nei salti}
Le etichette possono essere utilizzate all' interno dei salti per non dover calcolare manualmente gli offset e gli indirizzi:
\begin{verbatim}
    RESET:
        RJMP START
    
    START:
        LDI r16, 0x5f
        LDI r17, 0x08
\end{verbatim}
Il codice di un ciclo di ritardo potrebbe quindi essere scritto:
\begin{verbatim}
        LDI r16, 100
    CICLO:
        DEC r16
        BRNE CICLO
\end{verbatim}

\subsubsection{Etichette nelle espressioni}
Le etichette possono essere utilizzate, come placeholder dei loro valori, all' interno delle espressioni, purché se ne conosca il valore a tempo di assemblaggio.







