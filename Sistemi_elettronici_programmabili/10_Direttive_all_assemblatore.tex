\section{Altre strutture dell' assemblatore}
\subsection{Direttive}
Abbiamo detto che l' assemblatore si occupa di prendere il nostro codice assembly e produrre un file caricabile sulla flash del controllore, tuttavia non fa solo questo.
In particolare si occupa anche di gestire gli indirizzi attraverso le etichette e ci toglie il lavoro di fare i calcoli per i salti relativi.

Al suo interno ha 3 location\_counter:
\begin{itemize}
    \item CSEG: location\_counter della flash
    \item DSEG: location\_counter della SRAM
    \item ESEG: location\_counter della EEPROM
\end{itemize}
questi 3 contatori sono usati per mantenere gli indirizzi mentre si avanza con la traduzione, all' avvio partono tutti da 0 anche se DSEG a 0 è inutile in quanto lì non c'è SRAM ed all' avvio è attivo il contatore CSEG quindi di default siamo in una sezione codice.

Possiamo dare delle istruzioni all' assemblatore attraverso le \emph{direttive}, si usano per modificare i contatori, aggiungere informazioni utili e tanto altro.
Iniziano tutte con il . davanti.

\subsubsection{.BYTE}
Ci permette di creare della "variabili" cioè associare un nome mnemonico ad un indirizzo della SRAM.
\begin{verbatim}
    NOME_VARIABILE: .BYTE <dimensione>
\end{verbatim}
il nome viene specificato tramite una etichetta, successivamente la direttiva .BYTE alloca tanti byte quanti specificati nella dimensione.
\begin{verbatim}
    array10: .BYTE 10
    array20: .BYTE 10+10
\end{verbatim}
Dato che stiamo parlando della SRAM l' assemblatore usa il contatore DSEG per assegnare l' indirizzo, quindi lo incrementa.

\subsubsection{.DSEG, .CSEG, .ESEG}
Sono direttive che indicano all' assemblatore quale location counter usare come principale da quel momento in poi.
Di fatto usarne uno indica che si sta creando una nuova sezione in quel tipo di memoria.

\subsubsection{.DB}
Si usa per allocare costanti, quindi diamo un nome ad un valore scritto nella FLASH.
\begin{verbatim}
    NOME_COSTANTE: .DB <lista_di_valori>
\end{verbatim}
in base alla lunghezza della lista sceglie quanti byte utilizzare e quindi quanto lunga sarà questa struttura.
Essendo in FLASH alloca sempre un numero pari di byte per prendere tutta la riga.
Se specifichiamo un valore più grande di 255 esso verrà troncato ai primi 8 bit.
\begin{verbatim}
    consts: .DB 0, 255, 0b01010101, -128, 0xaa
    str:    .DB "asg!\0"
    char:   .DB 'c'
\end{verbatim}

\subsubsection{.DEF}
Si usa per dare un sinonimo ad un registro.
\begin{verbatim}
    .DEF Symbol=Register
\end{verbatim}

\subsubsection{.DEVICE}
Si usa per specificare per quale dispositivo si sta scrivendo il codice.
Questo è utile per far comprendere all' assemblatore quale subset di istruzioni sono accettabili.
Non è necessario usarlo, in tal caso però l' assemblatore userà tutte le istruzioni possibili quindi se una non è supportata dal microcontrollore non ci verrà detto.

\subsubsection{.DW}
Si usa per allocare costanti a 16 bit.
\begin{verbatim}
    NOME_COSTANTE: .DW <lista_di_valori>
\end{verbatim}
Anche in questo caso se la variabile è più di 16 bit verrà troncata.
\begin{verbatim}
    varlist: .DW 0, 0xffff, 0b1001100110011001, -32768, 65535
\end{verbatim}

\subsubsection{.EQU}
Definisce un simbolo con un valore.
\begin{verbatim}
    .EQU var1=3
    .EQU var2=var1*3
\end{verbatim}
Si possono usare tutte le espressioni costanti per indicare il valore.

\subsubsection{.EXIT}
Si usa per indicare la fine di un file, quando viene incontrata dall' assemblatore esso smette di processare il file corrente.
Non è obbligatorio usarla.

\subsubsection{.INCLUDE}
Include il file specificato nel punto in cui uso la direttiva.
In questo assembly, detto \emph{assoluto} non conviene separare il codice in troppi file per lavorare in gruppo, questo perché le etichette non possono ripetersi pertanto diverse persone a scrivere il codice significa più probabilità di commettere errori di questo tipo.
Gli assembly che permettono questa feature sono detti \emph{relativi}.

\subsubsection{.LIST}
Abilita la creazione del list file, cioè un file che contiene gli indirizzi delle singole istruzioni, il loro codice operativo assieme all' assembly.
Può tornare utile ai fini di debug.

\subsubsection{.LISTMAC}
Aggiunge le macro nel listfile.

\subsubsection{.MACRO}
Si usa per creare delle pseudoistruzioni cioè mnemonici che poi vengono sostituiti con altre operazioni.
\begin{verbatim}
    .MACRO NOME_MACRO
        <espanzione della macro>
    .ENDMACRO
\end{verbatim}
All' interno della espansione si possono usare i parametri usando la sintassi \verb{@n{ con $n$ il numero della posizione del parametro.
\begin{verbatim}
    .MACRO SUBI16
        subi @1, low(@0)
        sbci @2, high(@0)
    .ENDMACRO
\end{verbatim}
Per usare la macro non serve altro che scriverne il nome e specificare i parametri.
\begin{verbatim}
    .CSEG
    SUBI16 0x1234, r16, r17
    ; verrà tradotto con
    ; subi r16, low(0x1234)
    ; sbci r17, high(0x1234)
\end{verbatim}

Si consiglia di usare le macro per espansioni di poche istruzioni o pseudoistruzioni, se sono tante ed usate spesso conviene creare ed usare delle sub-routine.

\subsubsection{.ORG}
Altera a mano il location\_counter attivo in quel momento


\subsection{Funzioni}
Ci sono anche alcune funzioni che si possono applicare ai valori costanti, queste vengono calcolate dall' assemblatore che poi sostituisce le chiamate con i valori effettivi.

\subsubsection{LOW}
Applicata ad un numero restituisce gli 8 bit più bassi.

\subsubsection{HIGH}
Applicata ad un numero restituisce i bit dall'8 al 15.

\subsubsection{EXP2}
Applicata ad un numero restituisce la sua potenza di 2.

\subsubsection{LOG2}
Applicata ad un numero restituisce la parte intera del suo logaritmo in base 2.


\subsection{Operatori}
Abbiamo a disposizione tutti gli operatori del C per calcolare le espresioni costanti.
\begin{table}[H]
    \centering
    \begin{tabular}{c|c}
        Operatore & Significato \\
         ! & NOT logico \\
         $ \Tilde{} $ & NOT binario \\ 
         - & meno unario \\ 
         * & moltiplicazione \\ 
         / & divisione intera \\ 
         + & somma \\ 
         - & sottrazione \\ 
         $<<$ & shift a sinistra \\
         $>>$ & shift a destra \\
         $<$ & minore \\
         $<=$ & minore o uguale \\
         $>$ & maggiore \\
         $>=$ & maggiore o uguale \\
         == & controllo di uguaglianza \\
         != & controllo di diversità \\
         \& & AND bit a bit \\
         $\wedge$ & XOR bit a bit \\
         $|$ & OR bit a bit \\
         \&\& & AND logico \\
         $||$ & OR logico \\
    \end{tabular}
\end{table}



