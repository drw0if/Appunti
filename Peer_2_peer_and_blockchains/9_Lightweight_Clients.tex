Blockchain systems tend to grow as they are an append only ledger.
A blockchain client to interact with the system can:
\begin{itemize}
    \item download all the blockchain, storing and verifying the entire blockchain history;
    \item forfeit blockchain security guarantees and place its trust on third party intermediary servers.
\end{itemize}
In the recent years an alternative was developed: \emph{light client}.
Using this concept a resource constrained client (browsers, mobile phones, embedded devices, etc) can participate in the system by querying and/or submitting transactions without holding the full blockchain, while still having the blockchain's security guarantees.

\section{Bitcoin lightweight client}
SPV (Simplified Payment Verification) or lightweight client downloads only the header of the blocks (approximately 1000 times smaller, because block header is only 80 bytes).
This way the client is still able to verify PoW because the header contains the nonce and also the validity of a transaction.

A SVP node can talk to a full node in order to download the header-only blockchain, then the client builds a bloom filter with only the addresses in the wallet and sends it to the full node.
Then the full node exchanges to the light client the headers of the new blocks, not already downloaded by SVP, and also transactions involving requested addresses, also a Merkle tree paths in order to solve a Merkle proof and be sure that the transaction is inside the block.

NB: the idea of bloom filter is both for space occupancy and for privacy, because the client does not necessarily transmit it's addresses.





