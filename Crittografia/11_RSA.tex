\section{RSA}
\subsection{Generazione delle chiavi}
Partiamo dalla creazione della coppia di chiavi, spetta al destinatario:
\begin{itemize}
    \item Si scelgono due primi $p$, $q$ molto grandi ($\approx$1000 bit) (adesso si consiglia che $p \cdot q \approx 2048$ bit). Vanno generati in tempo polinomiale.
    \item Si calcolano $n = p \cdot q$ e $\phi(n) = (p-1) \cdot (q-1)$. Anche qui operazioni in tempo polinomiale nella grandezza dell'input.
    \item Si sceglie $e < \phi(n)$ tale che: $MCD(e, \phi(n))=1$
    \item Si calcola $d = e^{-1} \mod \phi(n)$. Anche questo si può calcolare in tempo polinomiale usando Euclide Esteso.
    
    NB: essendo $MCD(e, \phi(n))=1$ esiste unico $e^{-1}$.
\end{itemize}

Alla fine si hanno quindi le due chiavi:
$$ K_{pub} = (e,n) $$
$$ K_{priv} = (d,n) $$

Naturalmente va tenuto segreto anche il resto delle informazioni: $p$, $q$, $\phi(n)$.

\subsection{Cifratura e decifratura}
I messaggi sono sequenze binarie trattate come un intero! Lo indicheremo con $m$ e deve verificare $m < n$ perché non sarebbe decifrabile altrimenti.
Se $m \geq n$ lo dividiamo in blocchi di $b = \lfloor \log_2n \rfloor$ bit e si aggiunge una dipendenza tra un blocco e l'altro.
Solitamente si fissa un limite comune per la dimensione dei blocchi, prevenendo così l'uso di cifrari poco sicuri $m < 2^b < n$.

La cifratura è quindi eseguita:
$$ c = C(m, K_{pub}) = m^{e} \mod n$$

La decifratura è invece eseguita:
$$ m = D(c, K_{priv}) = c^{d} \mod n$$

Si possono eseguire in tempo polinomiale tramite la fast-exp.

\subsection{Dimostrazione della correttezza}
Si vuole dimostrare:
$$ D(C(m, K_{pub}), K_{priv}) = m $$

Possiamo analizzare 3 casi distinti:

\subsubsection{Caso 1}
Se $p$ e $q$ non dividono $m$, cioè se $m$ ed $n$ sono co-primi. In questo caso vale $MCD(m,n)=1$.

Se componiamo le formule per criptare e decriptare otteniamo:
$$ (m^e \mod n)^d \mod n = m^{e \cdot d} \mod n $$
ricordando che $ed \equiv 1 \mod \phi(n)$ applicando la definizione di congruenza in modulo possiamo scrivere: $ed = 1 + r \cdot \phi(n)$ per qualche $r \in \mathbb{N}$.
Possiamo quindi scrivere:

$$ m^{ed} \mod n = m^{1 + r \cdot \phi(n)} \mod n = m \cdot (m^{\phi(n)})^{r} \mod n$$

Essendo $m$ ed $n$ coprimi vale il Teorema di Eulero: $m^{\phi(n)} = 1 \mod n$ quindi:
$$ m \cdot 1^r \mod n = m \mod n = m $$

\subsubsection{Caso 2}
Supponiamo che $m$ sia divisibile per $p$ e non per $q$. Allora sarà vero che:
$$ m^{ed} = m^{ed} - m \equiv 0 \mod p $$

Ricaviamo una relazione simile anche per $q$ applicando la definizione di riduzione in modulo:
$$ m^{ed} \mod q = m^{\phi(n) \cdot r + 1} \mod q $$

Ora ricordiamo la definizione $\phi(n) = (p-1)\cdot(q-1)$ e possiamo riscrivere come:
$$ m^{ed} \mod q = m^{(p-1)\cdot(q-1)\cdot{r} + 1} \mod q = (m^{q-1})^{(p-1)\cdot{r}}\cdot{m} \mod q $$

Applicando il teorema di Eulero anche qui otteniamo:
$$ 1^{(p-1)\cdot{r}}\cdot{m} \mod q = m \mod q$$

Possiamo quindi scrivere:
$$ m^{ed} \equiv m \mod q \implies m^{ed} - m \equiv 0 \mod q $$
Abbiamo quindi ottenuto:
\begin{equation}
    \begin{cases}
    $$ m^{ed} - m \equiv 0 \mod q $$ \\ 
    $$ m^{ed} - m \equiv 0 \mod p $$
    \end{cases}
\end{equation}
che significa che $ m^{ed} -m $ è divisibile sia per $p$ che per $q$ quindi lo sarà anche per $n = p \cdot q$, possiamo quindi dire:
$$ m^{ed} -m \equiv 0 \mod n => m^{ed} \equiv m \mod n = m $$

\subsubsection{Caso 3}
Avere $m$ divisibile sia per $p$ che per $q$ significa che è divisibile per $n$, il che è impossibile in quanto un messaggio, per essere corretto, deve soddisfare $m < n$.

\subsection{Sicurezza ed attacchi}
La sicurezza è legata alla fattorizzazione, se fattorizzo $n$ ho $p$ e $q$ e da li posso trovare $\phi(n)$. Non sappiamo però se è necessaria la fattorizzazione o si potrebbe forzare in altri modi.

Potrebbe essere necessaria perché il problema è quello di calcolare $m = \sqrt[e]{c} \mod n$ ed è difficile quanto fattorizzare $n$.
Calcolare $\phi(n)$ direttamente da $n$ equivale a fattorizzare $n$.
In particolare il calcolo di $\phi(n)$ o la fattorizzazione di $n$ si trasformano l'uno nell'altro in tempo polinomiale: sono computazionalmente equivalenti:

Fattorizzato $n = p \cdot q \xrightarrow{} \phi(n) = (p-1) \cdot (q-1)$

Calcolato $\phi(n) \xrightarrow{} $
$$ \phi(n) = (p-1) \cdot (q-1) $$
$$ = p \cdot q - (p+q) + 1 = n - (p+q) + 1 $$

calcolo quindi $ x_1 = p+q = n - \phi(n) + 1 $

poi $ (p-q)^2 = (p+q)^2 - 4 \cdot n = (x_1)^2 - 4 \cdot n $

quindi $ x_2 = p - q = \sqrt{x_1^{2} - 4 \cdot n}$

dato che conosco
\begin{equation}
    \begin{cases}
    x_1 = p + q \\ 
    x_2 = p - q
    \end{cases}
\end{equation}

posso quindi trovare:
\begin{equation}
    \begin{cases}
    p = \frac{x_1 + x_2}{2} \\
    q = \frac{x_1 - x_2}{2}
    \end{cases}
\end{equation}

Un altro attacco che posso fare è cercare di brutare $d$ partendo da $(n,e)$ ma anche questo sembra costoso quanto fattorizzare $n$.

NB: fattorizzare non è $NP-hard$ quindi non si esclude che esista un algoritmo efficiente. Sulla macchina quantistica infatti esiste l'algoritmo di \emph{Schorr}.

\clearpage

\subsection{Fattorizzazione di n}
La fattorizzazione è difficile ma sempre un po' meno:
\begin{itemize}
    \item l'hardware migliora e gli algoritmi si affinano
    \item esistono algoritmi di complessità sub-esponenziale come il \emph{General Number Field Sieve} (GNFS) che richiede O($2^{\sqrt{b \cdot log b}}$) con $b$ la dimensione di $n$. Un attacco brute-force è O($2^b$)
    \item con la potenza di calcolo attuale usando GNFS si fattorizza fino a 768 bit
    \item per interi con una certa struttura esistono algoritmi di fattorizzazione efficienti
    \item fattorizzazione e logaritmo discreto non sono $NP-hard$ quindi sui computer quantistici performano bene
\end{itemize}

\subsection{Scelta ottimale dei parametri}
Per avere una sicurezza come quella che si ottiene con AES a 128 bit si dovrebbe ricorrere a $n$ con circa 3072 bit, si sale addirittura a 15360 bit di modulo se lo si compara con AES256.
$p$ e $q$ vanno quindi scelti \emph{molto} grandi, \emph{entrambi} (attorno ai 1700 bit).

Sia $p-1$ che $q-1$ devono contenere fattori grandi (altrimenti $n$ si fattorizza velocemente).

$MCD(p-1, q-1)$ deve essere piccolo! Conviene scegliere $p$ e $q$ tali che $\frac{p-1}{2}$ e $\frac{q-1}{2}$ siano coprimi.

Mai riutilizzare uno dei primi per altri moduli:
\begin{equation}
    \begin{cases}
    n_1 = p \cdot q_1 \\
    n_2 = p \cdot q_2
    \end{cases}
    \implies MCD(n_1, n_2) = p
\end{equation}

Scegliere $p$ e $q$ distanti tra loro! Se $p \approx q$ allora $n \approx q^2 \approx p^2$, quindi $p \approx \sqrt{n}$ e si cerca nei dintorni. Valgono infatti:
$$ (\frac{p+q}{2})^2 -n = (\frac{p-q}{2})^2 $$

Mi dice che $\frac{p+q}{2} > \sqrt{n}$ perché a dx è sempre $> 0$ se $p \neq q$.

Inoltre $\frac{(p+q)^2}{4}-n$ è un quadrato perfetto: $(\frac{p-q}{2})^2$.

Posso quindi cercare tra gli interi maggiori di $\sqrt{n}$ fino a trovare $z$ tale che $z^2 - n = w^2$.

Supponiamo dunque di averli trovati:
\begin{equation}
    \begin{cases}
    z = \frac{p+q}{2} \\
    w = \frac{p-q}{2}
    \end{cases}
\end{equation}

concludiamo quindi:
\begin{equation}
    \begin{cases}
    p = z+w \\
    q = z-w
    \end{cases}
\end{equation}


\subsection{Attacchi con esponenti bassi}
Valori di $e$ e $d$ bassi portano ad accelerare nell'algoritmo, tuttavia se $d$ è piccolo posso brutarli.
Inoltre se $m$ è piccolo ed anche $e$ lo è, può succedere che $m^e < n$ quindi non ci sia la riduzione in modulo. In tal caso decripto semplicemente calcolando $\sqrt[e]{m}$.

\subsection{Attacchi a tempo}
Si basano sul tempo di esecuzione dell'algoritmo di decifrazione.
Si può quindi determinare $d$ analizzando il tempo impiegato per decifrare.
Nell'algoritmo delle quadrature successive infatti si esegue una moltiplicazione ad ogni iterazione più una ulteriore moltiplicazione modulare per ciascun bit uguale ad 1 in $d$.
Guardando il tempo quindi si può stimare il numero di bit ad 1 in $d$.
Si può risolvere aggiungendo un ritardo casuale alla fine.
Ne soffre sia RSA che DH.

\subsection{Attacchi sulla scelta di \texorpdfstring{$e$}{e}}
Alcune scelte non portano modifiche nel messaggio:
$$ e \neq \frac{\phi(n) + 2}{2} $$
dato che 2 divide sia $(p-1)$ che $(q-1)$ si ha che: $ m^e \mod n = m $
quindi non si ha una vera cifratura.

Vale inoltre:
$$ e \neq \frac{\phi(n) + k}{k} $$
per qualsiasi $k$ che divide $(p-1)$ e $(q-1)$ con $m$ ed $n$ coprimi.
Anche in questo caso infatti $ m^e \mod n = m $.

Un altro attacco possibile se $e$ è troppo piccolo:
\begin{itemize}
    \item supponiamo che ci siano $e$ utenti che hanno scelto lo stesso valore piccolo di $e$
    \item gli $e$ utenti ricevono lo stesso messaggio $m$
\end{itemize}
si hanno dunque:
\begin{equation}
    \begin{cases}
    c_1 = m^e \mod n_1 \\
    c_2 = m^e \mod n_2 \\
    ... \\
    c_e = m^e \mod n_e \\
    \end{cases}
\end{equation}
sarà quindi vero che $m < n_i \forall i: 1 \leq i \leq e$.
Ipotiziamo che $n_1$, $n_2$, $\_$, $n_e$ siano coprimi tra loro (se non lo fossero si potrebbe fattorizzarne qualcuno usando il $MCD$).
Per il teorema cinese del resto esiste e si può calcolare un unico $m'$ tale che:
\begin{equation}
    \begin{cases}
    m' < n = n_1 \cdot n_2 \cdot ... \cdot n_e \\
    m' \equiv m^e \mod n
    \end{cases}
\end{equation}

Dato che $m < n_i$ so che $m \cdot m \cdot ... \cdot m = m^e < n_1 \cdot n_2 \cdot ... \cdot n_e = n$.
Quindi la riduzione in modulo non avviene e posso calcolare $m = \sqrt[e]{m'}$.
Se voglio continuare ad usare $e$ piccolo posso aggiungere un padding a caso ad ogni utente in modo che i messaggi non siano più tutti uguali.

\subsection{Attacco con lo stesso valore di n}
Supponiamo che due utenti abbiano generato una chiave pubblica con lo stesso $n$ ma diversi $e$:
$$ <e_1,n>, <e_2,n> $$
Supponiamo inoltre che $mcd(e_1, e_2) = 1$ allora $\exists r,s \in \mathbb{Z}$ tale che $e_1 \cdot r + e_2 \cdot s = 1$ e li possiamo trovare utilizzando Euclide Esteso in quanto sono nella forma $ax +by = mcd(a,b)$.
Saranno allora $r<0, s>0$.

Si intercetta $c_i = m^{e_i} \mod n$ e si voglia trovare $m$:
$$ m = m^1 = m^{r \cdot e_1 + s \cdot e_2 } = (m^{e_1} \mod n)^{r} \cdot (m^{e_2} \mod n)^{s} = $$
$$ = (c_1^r \cdot c_2^s) \mod n = ((c_1^{-1})^{-r} \cdot c_2^s) \mod n $$

Si calcola quindi $c_1^{-1} \mod n$ che esiste se $c_1$ ed $n$ sono coprimi (se non lo fossero potremmo fattorizzare $n$). Si calcola quindi $m = (c_1^{-1})^{-r} \cdot (c_2)^{s} \mod n$ il tutto fattibile in tempo polinomiale.

\subsection{Cifrari ibridi}
Il principale utilizzo della crittografia a chiave pubblica non è quello di criptare la comunicazione ma quello di scambiare le chiavi! Alice e Bob vogliono parlare tra di loro in maniera sicura:
\begin{itemize}
    \item Alice: 
    \begin{itemize}
        \item sceglie una chiave AES (detta \emph{chiave di sessione})
        \item la cifra con la chiave pubblica di Bob
    \end{itemize}
    \item Bob: decifra il crittogramma in quanto in possesso della chiave privata
\end{itemize}
Ora la comunicazione può avvenire tramite AES sempre in sicurezza ma soprattutto in velocità

NB: al termine della sessione tuttavia la chiave AES va cancellata e ne va creata una nuova ad ogni sessione di comunicazione

Un problema di questo approccio è che l'onere di creare la chiave di sessione sta ad Alice. Bob non può essere sicuro che la chiave sia stata generata in modo corretto!
In genere sarebbe preferibile porre ambo le parti sullo stesso livello.