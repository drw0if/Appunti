\section{Generatori pseudocasuali}

\subsection{Sorgente binaria casuale}
Genera una sequenza di bit:
\begin{itemize}
    \item $P(0) = P(1) = \frac{1}{2}$ ma si può indebolire richiedendo $P(0)>0, P(1)>0$ e immutabili durante il processo di generazione
    \item La generazione di un bit è indipendente da quella di altri bit: non si può prevedere il valore di bit osservando quelli già generati
\end{itemize}
Perché possiamo indebolire la prima richiesta? Supponiamo che $P(0) > P(1)$ e si generi la sequenza:
$$ 00|11|00|11|10|00|01|01|00 $$
elimino le sottosequenze uguali ed associo ad ogni coppia un valore 0 o 1:
$$ 01 \implies 0 $$
$$ 10 \implies 1 $$
ottenendo quindi:
$$ 100 $$
Questa sequenza ottenute sarà la nostra sequenza casuale.

\subsubsection{Esistono vere sorgenti casuali?}
Non lo sappiamo perché la fisica ci dice che ogni cosa lascia nello spazio e nel tempo dei rimasugli quindi non sembra essere ottenibile perfettamente.
Quello che si fa è sfruttare dei fenomeni casuali naturali.

\subsection{Generazione di sequenze brevi}
Vediamo alcune tecniche di generazione di sequenze brevi:
\begin{itemize}
    \item Fenomeni casuali presenti in natura (sorgenti di casualità tipo rumore del microfono)
    \item Processi software (come ad esempio la posizione della testina dell'hard disk o l'orologio del computer)
    \item Generazione mediante algoritmi matematici. Questo ultimo tipo costituisce i \emph{generatori di numeri pseudo-casuali}. Non sono ovviamente sequenze casuali secondo Kolmogorov in quanto prodotte da un programma breve
\end{itemize}

\subsection{Generatore}
E' un algoritmo che prende in input un \emph{seme} cioè una sequenza o un valore \emph{breve} e fornisce in output un \emph{flusso} di bit arbitrariamente lungo e periodico (cioè al suo interno contiene una sottosequenza che si ripete ogni tot periodo). Useremo quindi un periodo come sequenza casuale.

Un generatore è tanto migliore tanto più è lungo il periodo! Nella pratica un generatore è un \emph{amplificatore} di casualità.

Supponiamo 
\begin{itemize}
    \item $S$ : numero di bit del seme
    \item $n$ : lunghezza della sequenza ottenuta
\end{itemize}
avremo necessariamente che:
$$ 2^{S} << 2^{n} $$
con:
\begin{itemize}
    \item $2^{S}$: tutte le possibili sequenze diverse
    \item $2^{n}$: numero totale di sequenze diverse di lunghezza n
\end{itemize}

questi processi sono deterministici in quanto dato un seme la sequenza generata è sempre la stessa, quindi il vero numero di sequenze possibili da generare è $2^{S}$.

\subsection{Generatore lineare}
$$ x_{i} = \left( a \cdot x_{i-1} + b \right) \mod m $$
$$ a,b,m \in \mathbb{N} $$
$$ x_{0} = \text{seme} $$
I parametri vanno scelti in modo da poter generare tutti i valori da $0$ a $m-1$, quindi un periodo di lunghezza $m$. Inoltre appena $x_{i} = x_{0}$ la sequenza si ripete da capo. Per massimizzare il periodo ci servono:
\begin{itemize}
    \item $MCD(b, m) = 1$
    \item $(a-1)$ deve essere divisibile per ogni fattore primo di $m$
    \item $(a-1)$ deve essere un multiplo di 4 se $m$ lo è
\end{itemize}
Queste caratteristiche ci garantiscono dunque di ottenere una permutazione dei primi $m$ numeri.

Es:
$$ a=7, b=7, m=9, x_{0}=3 $$
$$ x_i = (7x_{i-1} + 7) \mod 9 $$
$$ \implies 3, 1, 5, 6, 4, 8, ... , 3 $$

NB: posso ottenere un generatore di sequenze binarie facendo $\frac{x_i}{m}$ e prendendo la parità della prima cifra decimale.

\subsection{Valutazione statistica}
Per studiare un generatore di sequenze pseudo-casuali possiamo effettuare dei test statistici valutando se la sequenza presenta le proprietà tipiche di una sequenza casuale:
\begin{itemize}
    \item \emph{test di frequenza}
    \item \emph{poker test}: sottosequenza di lunghezza fissa siano equamente distribuite
    \item \emph{test di autocorrelazione}: verifica che non ci siano regolarità nella sequenza
    \item \emph{run test}: verifica che le sottosequenze massimali (più lunghe possibili) abbiano una frequenza esponenziale negativa: più è lunga e meno è probabile trovarla con decrescita esponenziale
\end{itemize}

\subsection{Test di prossimo bit}
Per le applicazioni crittografiche si richiede anche il \emph{test di prossimo bit} (che implica anche i 4 test precedenti). Questo test controlla che sia impossibile fare previsioni sugli elementi della sequenza prima che siano stati generati. Se una sequenza passa questo test è impossibile fare previsioni sugli elementi della sequenza.

Un generatore binario supera il test di prossimo bit se $ \nexists $ un algoritmo polinomiale in grado di prevedere il bit $(i+1)$esimo della sequenza a partire dalla conoscenza degli $i$ bit precedentemente generati, con probabilità $> \frac{1}{2}$. Questi generatori si chiamano \emph{crittograficamente sicuri}.

\subsection{Generatore polinomiale}
E' un altro generatore non crittograficamente sicuro ma che passa i test statistici. E' nella forma:
$$ x_{i} = \left( a_{1}x_{i-1}^{t} + a_{2}x_{i-1}^{t-1} + ... + a_{t}x_{i-1} + a_{t+1} \right) \mod n $$

Si noti che è una versione generalizzata di quello lineare, può essere usato come generatore di sequenze binarie:

\begin{equation}
    \frac{x_{i}}{n} = r_{i} \longrightarrow
    \begin{cases}
        \text{prima cifra decimale pari} \implies 0 \\
        \text{prima cifra decimale dispari} \implies 1
    \end{cases}
\end{equation}

\subsection{Costruzione di generatori crittograficamente sicuri con funzioni one-way}
Una funzione \emph{one-way} è una funzione facile da calcolare ma difficile da invertire:
$$ x \xrightarrow{f(x)} y \text{ : tempo polinomiale} $$
$$ x \xrightarrow{f^{-1}(x)} y \text{ : tempo esponenziale} $$

Una idea potrebbe essere: data una funzione $f$ one-way e dato un seme $s = x_{0}$ calcolo la sequenza:
\begin{table}[!ht]
    \centering
    \begin{tabular}{c | c c c c c}
        $S$: & $x$ & $f(x)$ & $f(x_{1}) = f(f(x))$ & ... & $f^{(n)}(x) = f(x_{n-1})$ \\
         & $x_{0}$ & $x_{1}$ & $x_{2}$ & & $x_{n}$ \\
    \end{tabular}
\end{table}

si prende quindi il seme e si itera la funzione one-way un numero arbitrario di volte. Tuttavia se conosco x la sequenza è prevedibile! Se invece \emph{restituisco la sequenza al contrario} il problema si risolve! Se ti do $x_{i+1}$ non puoi calcolare facilmente $x_{i}$ ma solo $x_{i+2}$ che però ti ho già restituito precedentemente. Bisognerebbe passare per una funzione esponenziale ad ogni passo.

\subsection{Generatori binari crittograficamente sicuri}
Si usano \emph{predicati hard core} delle funzioni one-way: $b(x)$ è un predicato hard core per la funzione one-way $f(x)$ se:
\begin{itemize}
    \item $b(x)$ è facile da calcolare conoscendo $x$
    \item $b(x)$ è difficile da prevedere con probabilità $> \frac{1}{2}$ conoscendo solo $f(x)$
\end{itemize}
Es:
$$ x \xrightarrow{} f(x) = x^2 \mod n \text{ (n non primo)} $$
è una funzione one-way
$$ n = 77, x = 10, y = 10^{2} \mod 77 = 23 $$
il contrario prevederebbe una enumerazione di tutti i valori da 0 a $n-1$!
$$ b(x) = \text{x è dispari} \text{ è un predicato hard core} $$

\subsection{Generatore BBS}
Il generatore BBS (Blum-Blum-Shub), nato nel 1986 è un generatore crittograficamente sicuro! Si prendono $p$, $q$ primi (grandi) che verificano:

\begin{itemize}
    \item $ p \mod 4 = 3 $
    \item $ q \mod 4 = 3 $
    \item $ MCD(2 \lfloor \frac{p}{4} \rfloor + 1, 2 \lfloor \frac{q}{4} \rfloor + 1) = 1 $
\end{itemize}

si procede:

$$ n = p \cdot q $$
$$ y | MCD(y, n) = 1 $$

si calcola il seme $x_{0}$:

$$ x_{0} = y^{2} \mod n $$

si calcola una successione di $ m \leq n $ interi con:

$$ x_{i} = (x_{i-1})^{2} \mod n $$
$$ b_{i} = 1 \Longleftrightarrow x_{n-1} \text{ è dispari} $$

si hanno quindi:

\begin{table}[!ht]
    \centering
    \begin{tabular}{c c c}
        $b_{0} = 1$ & $ \Longleftrightarrow $ & $ x_{n} $ è dispari\\
        $b_{1} = 1$ & $ \Longleftrightarrow $ & $ x_{n-1} $ è dispari\\
        ... & ... & ... \\
        $b_{n} = 1$ & $ \Longleftrightarrow $ & $ x_{0} $ è dispari\\
    \end{tabular}
\end{table}

e si restituiscono in ordine: $b_{0}, b_{1}, ..., b_{n}$

I problemi di questo generatore sono: la lentezza, la necessità di numeri molto grandi e l'esecuzione di potenze di questi numeri già molto grandi. Ci sono altre tecniche usate, più veloci anche se meno sicure.

\subsection{Generatore di numeri pseudocasuali basati su cifrari simmetrici}
Si prendono un cifrario simmetrico ed una chiave, anziché usare un messaggio scegliamo un seme e produciamo la sequenza. Vediamo un esempio che sfrutta il DES ed è stato approvato dal FIPS (Federal Information Processing Standard):

\begin{verbatim}
    G = funzione di cifratura
    r = #bit delle parole prodotte (rDES = 64 bit)
    s = seme casuale di r bit
    m = #parole da produrre
    k = chiave segreta del cifrario

    Generatore(s, m): // flusso di output di m * r bit
        d = rappresentazione su r bit di data ed ora
        y = G(d, k)
        z = s
        for(i = 1; i <= m; i++):
            xi = G(y XOR z, k)
            z = G(y XOR xi, k)
            yield xi
\end{verbatim}