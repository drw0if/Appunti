\section{Identificazione, autenticazione e firma digitale}

L' \emph{identificazione} prevede che un sistema isolato o in rete debba essere in grado di accertare l'identità di un utente che richiede di accedere ai suoi servizi.

L' \emph{autenticazione} prevede che il destinatario di un messaggio possa essere in grado di accertare:
\begin{itemize}
    \item l'identità del mittente
    \item l'integrità del crittogramma ricevuto (che non sia stato modificato, che non sia stato sostituito)
\end{itemize}

NB: identificazione $\subset$ autenticazione

La \emph{firma digitale} ci permette invece:
\begin{itemize}
    \item il mittente non può negare di aver inviato il messaggio $m$: \emph{non ripudiabilità}
    \item il destinatario deve essere in grado di autenticate il messaggio: \emph{autenticazione}
    \item il destinatario non deve poter sostenere che $m' \neq m$ è il messaggio inviato dal mittente
\end{itemize}
In fine tutto questo deve essere verificabile da terzi!

Non sono modalità indipendenti, ciascuna estende le precedenti!
Sono funzionalità utilizzate per contrastare gli attacchi attivi.

Ci sono realizzazioni sia su cifrari simmetrici che asimmetrici, noi vedremo queste ultime versioni.

\subsection{Funzioni hash}
Una funzione hash: $f: X \xrightarrow{} Y$ è una funzione tale che:
$$ n = \mid X \mid >> m = \mid Y \mid $$

Essendo non iniettiva: $ \exists X_1, X_2, \_, X_m \subseteq X$ disgiunti tali che: 
$$ X = X_1 \cup X_2 \cup \_ \cup \_ X_m $$
$$ \forall i, x : x \in X_i : f(x) = y $$
ogni insieme $X_i$ è l'insieme di collisioni e si vuole che le dimensioni dei vari $X_i$ sia più omogenea possibile.
Quindi due elementi estratti a caso da $X$ hanno probabilità circa $\frac{1}{m}$ di avere la stessa immagine $y$.

Inoltre si vuole che immagini \emph{simili} tra loro appartengano a sottoinsiemi diversi.
Bisogna gestire le collisioni.
Queste sono le richieste per una buona funzione di hash

\subsection{Funzioni hash one-way}
Per le applicazioni crittografiche si devono avere le seguenti proprietà:
\begin{itemize}
    \item $\forall x \in X$ è computazionalmente facile calcolare $y = f(x)$
    \item one-way: per la maggior parte degli $y \in Y$ è computazionalmente difficile determinare $x \in X$ tale che: $f(x) = y$
    \item \emph{claw-free}: è computazionalmente difficile determinare $<x_1, x_2>$ in $X$ tale che: $f(x_1)=f(x_2)$ 
\end{itemize}

\subsubsection{MD5 (Message Digest v5)}
Si tratta di una famiglia di algoritmi.
Sono stati pubblicati MD2, MD4, MD5.
Nei primi due furono trovate falle e Rivest propose MD5 nel 1992.

Riceve in input un sequenza $S$ di 512 bit e produce un'immagine di 128 bit.
Si \emph{digerisce} la sequenza rudicendone la lunghezza ad $\frac{1}{4}$.

Non resiste alle collisioni e nel 2004 è uscito di scena con debolezze varie.

Si considera severamente compromesso (anche dallo stesso Rivest).

Si usa ancora in altri contesti non crittografici.

\subsubsection{RIPEMD-160}
E' la versione matura dell'MD*. Nata nel 1995 produce immagini di 160 bit e non presenta i difetti di MD5.

\subsubsection{SHA (Secure Hash Algorithm)}
Progettata dal NIST ed NSA nel 1993.
Opera su sequenze lunghe fino a $2^{64}$ bit e produce immagini di 160 bit.
E' crittograficamente sicura.
La prima versione: SHA-0 conteneva debolezze e fu quindi rivista portando allo SHA-1.

Sono poi stati creati SHA-2: altri 4 algoritmi caratterizzati da digest più lunghi.

Nel 2007 a causa dei problemi di MD5 e SHA-0 il NIST ha richiesto nuovi standard e nel 2012 è stato selezionato un algoritmo che è diventano SHA-3, pubblicato ufficialmente nel 2015.

\subsubsection{SHA-1}
Opera su sequenze fino a $2^{64}-1$ bit e produce immagini di 160 bit.
E' largamente usata nei protocolli crittografici anche se non è più certificata come standard.

Opera su blocchi di 160 bit contenuti in un buffer di 5 registri da 32 bit ciascuno in cui sono caricati inizialmente dei valori pubblici.
Il messaggio $m$ viene poi concatenato con una sequenza di padding che ne rende la lunghezza multipla di 512 bit.

Il contenuto dei registri varia nel corso dei cicli successivi in cui questi valori ci combinano tra loro e con blocchi di 21 bit provenientei da $m$.
Alla fine i registri contengono SHA-1($m$).

\subsection{Identificazione su canali sicuri}
Supponiamo di avere un utente che vuole accedere alle risorse su un calcolatore.
Per identificarsi usa username e password.
Supponiamo che viaggiare su un canale sicuro, le uniche fonti di attacco sono quindi sul calcolatore remoto.
Per evitare problemi sul server remoto non si memorizzano le password in chiaro ma un loro hash.
In particolare sui sistemi UNIX quando un utente $U$ fornisce per la prima volta una password $P$ il sistema associa ad $U$ due sequenze binarie:
\begin{itemize}
    \item $S$: seme prodotto da un generatore pseudocasuale
    \item $Q = h(PS)$: hash della concatenazione di $P$ ed $S$
\end{itemize}

Ad ogni successiva connessione il sistema prende la password, la concatena ad $S$, ne esegue l'hash e controlla che sia uguale all'hash memorizzato.

In questa maniera un accesso al file delle password non rivela informazioni importanti.

\subsection{Protezione del canale}
Se il canale è insicuro la password può essere intercettata durante la sua trasmissione in chiaro.
Vediamo quindi un sistema di \emph{identificazione} basato su chiave pubblica e privata.

Per esempio siano $<e,n>$, $<d,n>$ le chiavi pubblica e privata di un utente $U$ che richiede l'accesso ai servizi offerti dal sistema $S$.
\begin{itemize}
    \item $S$ genera un numero casuale $r < n$ e lo invia in chiaro a $U$
    \item $U$ calcola:
    $$ f = r^d \mod n $$
    \emph{firma di $U$ su $r$}
    \item $S$ verifica la correttezza del valore ricevuto calcolando:
    $$ f^e \mod n $$
    e verificando l'uguaglianza. Se avviene l'identificazione ha successo
\end{itemize}

Se $S$ non è trusted potrebbe inviare un $r$ particolare, anziché randomico, per cercare di carpire informazioni sulla chiave privata!

Si tratta di invertire le operazioni di crittazione e decifrazione rispetto all' RSA, ciò è possibile poiché è commutativo:
$$ (x^e \mod n)^d \mod n = (x^d \mod n)^e \mod n = x $$

Inoltre $f$ può essere generata solo da $U$ che possiede $<d, n>$.

Per eseguire l'autenticazione invece si usa il MAC: \emph{message authentication code}.
\begin{itemize}
    \item mittente e destinatario concordano una chiave segreta $k$
    \item il mittente allega al messaggio un MAC:
    $$ A(m,k) $$
    allo scopo di garantire la provenienza e l'integrità del messaggio.
    Spedisce quindi $<m, A(m,k)>$ in chiaro oppure spedisce $<C(m,k'),A(m,k)>$ frutto di una cifratura
    \item il destinatario entra in possesso di $m$, essendo a conoscenza di $A$ e $k$ ricalcola $A(m,k)$ e ne confronta il valore con quello inviato dal mittente. Se i MAC corrispondono allora ha successo
\end{itemize}

\subsection{MAC}
Deve essere una immagine breve del messaggio che si può ottenere solo se si è a conoscenza del valore comune $k$.
Ci sono varie implementazioni basat su cifrari asimmetrici, simmetrici e funzioni hash one-way.
In particolare la versione one-way prevede:
$$ A(m,k) = h(mk) $$
con $h$ una funzione hash one-way.

E' sicuro in quanto $k$ viaggia nel MAC ed anche se venisse scoperto non è invertibile in maniera semplice.
Non può nemmeno sostituire tutto in quanto dovrebbe corredare il messaggio di un nuovo MAC che però non può calcolare essendo allo scuro del valore di $k$.

NB: se si usa un cifrario a blocchi in modalità CBC si può usare l'ultimo blocco come MAC in quanto è funzione dell'intero messaggio.

\subsection{Firma manuale}
\begin{itemize}
    \item E' autentica e non falsabile: prova che chi la ha prodotta è chi ha sottoscritto il documento
    \item Non è riutilizzabile: è legata al documento su cui è stata apposta
    \item il documento non è alterabile: chi ha prodotto la firma è sicuro che questa si riferirà solo al documento sottoscritto nella sua forma originale
    \item non può essere ripudiata da chi l'ha apposta costituendo prova legale di un accordo o dichiarazione
\end{itemize}

\subsection{Firma digitale}
Non può banalmente essere la digitalizzazione di un documento originale firmato manualmente (si potrebbe banalmente tagliare ed incollare su un altro documento).
Deve quindi avere una forma che dipenda dal documento per essere inscindibile.
Ci sono protocolli che si basano sia su cifrari simmetrici che asimmetrici

\subsection{Protocollo 1: messaggio in chiaro e firmato (DH)}
L'utente $U$ dispone di una $K_{upriv}$ e $K_{upub}$. $C$ e $D$ sono funzioni di cifratura e decifratura di un cifrario asimmetrico.
La firma funziona:
\begin{itemize}
    \item $U$ genera la firma 
    $$f = D(m, k_{upriv})$$
    \item spedisce all'utente $<U,m,f>$
    \item l'utente riceve $<U,m,f>$ e calcola 
    $$C(f, K_{upub}) = m$$
    e verifica che il risultato sia proprio $m$
\end{itemize}

L'indicazione del mittente $U$ consente a $V$ di selezionare la $K_{upub}$ da utilizzare.
Per far funzionare questo meccanismo è importante che $C$ e $D$ siano commutative.

Questo protocollo soddisfa i requisiti della forma digitale:
\begin{itemize}
    \item è autentica e non falsabile in quanto $K_{upriv}$  è nota solo al destinatario, per falsificare la firma bisogna conoscere $K_{upriv}$ ma è difficile
    \item il documento non può essere alterato in quanto $m$ e $f$ sarebbero inconsistenti
    \item $U$ non può ripudiare la firma perché può averla prodotta solo lui
    \item la firma non è riutilizzabile in quanto $f$ è immagine di $m$
\end{itemize}

La verifica la possono fare tutti, tuttavia non ha nessuna cifratura del messaggio in quanto pur cifrando il messaggio posso ricavarlo dalla firma.

\subsection{Protocollo 2: messaggio cifrato e firmato}
\begin{itemize}
    \item $U$ genera la firma:
    $$ f = D(m, K_{upriv}) $$
    per $m$
    \item si calcola il crittogramma firmato:
    $$ c = C(f, K_{vpub}) $$
    con la chiave del destinatario
    \item invio $<U, c>$
    \item $V$ riceve la coppia $U, c$ e la decifra:
    $$ D(c, K_{vpriv}) = f $$
    \item verificando la firma si ottiene anche il messaggio:
    $$ C(f, K_{upub}) = C(D(m, K_{upriv}), K_{upub}) = m $$
\end{itemize}

Ci servono quindi:
$$ <d_u, n_u>, <e_u, n_u> \textbf{ chiavi di U} $$
$$ <d_v, n_v>, <e_v, n_v> \textbf{ chiavi di V} $$

L'utente U:
\begin{itemize}
    \item firma il messaggio: $f = m^{d_u} \mod n_u$
    \item cifra il messaggio: $c = f^{e_v} \mod n_v$
    \item spedisce la coppia: $<U, c>$
\end{itemize}

L'utente V:
\begin{itemize}
    \item riceve e decifra: $f = c^{d_v} \mod n_v$
    \item decifra $f$ con $k_{upub}$: $f^{e_u} \mod n_u = m$
\end{itemize}

Se $m$ è significativo concludo che è autentico.

Per la correttezza del procedimento èp necessario che $n_u \leq n_v$ affinché sia vero che $f < n_v$ e quindi $f$ possa essere cifrato correttamente.
In questo modo tuttavia $V$ non può rispondere in quanto dovrebbero avere $n_u = n_v$ ma è poco sicuro e probabile.

Possiamo ovviare quindi dando ad ogni utente due chiavi distinte: una per la firma $< H$ ed una per la cifratura $> H$ con $H$ pubblico e molto grande.

NB: si può attaccare se ci si procura la firma di un utente su un messaggio apparentemente privo di senso

\subsubsection{Attacco 1}
Supponiamo che un utente $U$ invii una risposta automatica (ack) al mittente ogni volta che riceve un messaggio $m$, e che questo messaggio sia il crittogramma della firma di $U$ su $m$.
Un crittoanalista attivo $X$ può decifrare i messaggi inviati a U:
\begin{itemize}
    \item $X$ intercetta $c$ firmato e crittografato inviato da $V$ a $U$, lo rimuove dal canale e lo rispedisce a $U$ con identificativo $X$. (cancella $<V,c>$ ed invia $<X,c>$)
    \item $U$ spedisce un ack ad $X$ in risposta
    \item $X$ usa l'ack per risalire al messaggio $m$ applicando le funzioni del cifrario con le chiavi pubbliche di $V$ e di $U$.
\end{itemize}

In particolare succede:
\begin{itemize}
    \item $V$ invia $c$ a $U$:
        \begin{equation}
            \begin{cases}
                c = C(f, K_{upub})
                f = D(m, K_{vpriv}        
            \end{cases}
            \implies <V,c>
        \end{equation}
    \item $X$ intercetta $c$, lo rimuove e rispedisce $<X,c>$
    \item $U$ decifra $c$:
    $$ f = D(c, K_{upriv}) $$
    e verifica la firma con la chiave pubblica di X ottenendo:
    $$ m' = C(f, K_{xpub}) $$
    \item $m' \neq m$ è privo di senso ma manda automaticamente l'ack $c'$ ad $X$:
    $$ f' = D(m', K_{upriv}) $$
    $$ c' = C(f', K_{xpub}) $$
    \item $X$ usa $c'$ per risalire ad $m$: decifra $c': D(c', K_{xpriv}) = f'$ e verifica $f': C(f', K_{upub}) = m'$. Da $m'$ ricava $f: D(m', K_{xpriv})=f$, verifica $f: C(f, K_{vpub}) = m$
\end{itemize}

Per evitare l'attacco si deve evitare l'ack automatico inviandolo solo se il messaggio ha senso.

\subsection{Protocollo resistente agli attacchi}
Si evita di firmare il messaggio!
Si crea un hash del messaggio (SHA) con una funzione hash one-way e vi si appone la firma sopra.
La firma non è quindi più soggetta a giochi algebrici.
\begin{itemize}
    \item il mittente $U$ calcola $h(m)$ e genera:
    $$ f = D(h(m), K_{upriv})$$
    \item calcola separatamente:
    $$ c = C(m, K_{vpriv}) $$
    \item spedisce a $V$ la tripla: $<U, c, f>$
\end{itemize}

La decifrazione e verifica invece:
\begin{itemize}
    \item $V$ riceve e verifica $<U, c, f>$
    \item decifra il crittogramma $c: m = D(c, K_{upriv})$
    \item calcola $h(m)$ e $C(f, K_{upub})$ e ne controlla l'uguaglianza, se sono uguali il messaggio è autentico
\end{itemize}
La firma si calcola più velocemente.

\subsection{Attacchi man-in-the-middle}
Le chiavi di cifratura sono pubbliche e non richiedono un incontro diretto per il loro scambio.
Un crittoanalista attivo può quindi intromettersi nella fase iniziale comportandosi come $V$ agli occhi di $U$ e come $U$ agli occhi di $V$.
\begin{itemize}
    \item $U$ richiede a $V$ la sua chiave pubblica
    \item $X$ intercetta la risposta contenente $K_{vpub}$ e la sostituisce con la sua chiave pubblica $K_{xpub}$
    \item $X$ si pone in ascolto in attesa dei crittogrammi spediti da $U$ a $V$ cifrati mediante $K_{xpub}$
    \item $X$ rimuove dal canale ciascuno dei crittogrammi, li decripta e li cifra con $K_{vpub}$ trafugata all'inizio, rispedendola a $V$
    \item $U$ e $V$ non si accorgono di nulla se il tutto è fatto velocemente
\end{itemize}

\subsection{Certification of Authority - CA}
Sono infrastrutture/enti che garantiscono la validità delle chiavi pubbliche e ne regolano l'uso gestendo la distribuzione delle chiavi a chi vuole comunicare.
La CA autentica l'associazione <utente, chiave pubblica> emettendo un certificato digitale.
Il certificato consiste della chiave pubblica e di una lista di informazioni relative al suo proprietario opportunamente firmate dalla CA.

Mantiene quindi un archivio accessibile a tutti ma protetto da scritture non autorizzate.
La chiave della CA è nota agli utenti che la mantengono protetta e la utilizzano per verificare la firma della stessa CA.

\subsection{Certificato digitale}
Si compone principalmente di:
\begin{itemize}
    \item indicazione del formato (numero di versione)
    \item nome della CA che lo ha rilasciato
    \item numero seriale che lo individua univocamente nella CA emittente
    \item specifica dell'algoritmo usato dalla CA per creare la firma digitale
    \item periodo di validità
    \item nome ed altre informazioni dell'utente a cui si riferisce il certificato
    \item nome dell'algoritmo, parametri e chiave pubblica usati dall'utente per cifratura e firma
    \item firma della CA su tutto quanto
\end{itemize}

Se $U$ vuole comunicare con $V$ può richiedere $K_{vpub}$ alla CA oppure direttamente a $V$ e poi lo convalida tramite la CA.

Dato che $U$ conosce $K_{CA-pub}$ può controllarne l'autenticità e la validità.
Se tutto va a buon fine la comunicazione parte.

Ci si può mettere nel mezzo solo falsificando la certificazione ma si assume che la CA sia fidata.

Esistono varia CA organizzate ad albero, la verifica quindi è più complicata in quanto si cerca il primo antenato comune tra le CA di $U$ e di $V$. Ogni utente poi mantiene in cache una copia dei certificati richiesti ed una copia di $K_{CA-pub}$ per non doverla richiedere.

\subsection{Protocollo 4: messaggio cifrato, firmato in hash e certificato}
Il mittente $U$:
\begin{itemize}
    \item si procura $cert$ di V

    \item calcola $h(m)$ e firma:
    $$ f = D(h(m),K_{upriv} $$

    \item calcola $c$:
    $$ c = C(m, K_{vpub}) $$

    \item spedisce <$cert_U$, c, f> ($cert_U$ contiene $K_{upub}$)
\end{itemize}

Il destinatario $V$:
\begin{itemize}
    \item riceve <$cert_U$, c, f> e verifica l'autenticità di $cert_U$ (e di $K_{upub}$) tramite la CA.

    \item decifra il crittogramma:
    $$ m=D(c, K_{vpriv}) $$

    \item verifica l'autenticità della firma:
    $$ c(f, K_{upub}) = h(m) $$
\end{itemize}

La catena ha un punto debole: i certificati non più validi, è quindi cruciale controllare periodicamente con la CA i certificati scaduti.